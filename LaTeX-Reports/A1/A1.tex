
\documentclass[12pt]{article}
\usepackage{graphicx} % for including images
\usepackage{hyperref} % for creating hyperlinks
\usepackage{amsmath} % for mathematical expressions
\usepackage{lipsum} % for generating dummy text

\begin{document}
% Title page information
\title{Raspberry Pi 4 Environment Configuration}
\author{Jaidon Lybbert, University of Washington}
\date{January 24, 2024}


% Generate the title page
\maketitle

% Abstract section
\begin{abstract}
This report documents the setup and configuration of a development environment for a remotely accessed Raspberry Pi 4 on a local WiFi network. The Raspbian-Lite OS is flashed to an SD card using Rufus on a Windows 10 laptop, then the Linux filesystem is directly modified on the SD card to set up SSH and enable connecting to the WiFi network, avoiding the need for a keyboard, mouse or monitor for first time set up.  Before doing any work, I describe my ideal environment based on my preferences, goals for the next 10 weeks, and available tools. Based on my experience, I layout my plan ahead of time and compile the resources I think I will need.
\end{abstract}

% Introduction section
\section{Introduction}
This document is organized as follows. In Section \ref{sec:background} I describe my experience with embedded systems and software, motivations and intent, and take a look at the physical interface to the Raspberry Pi 4. I devise a plan for setting up my preffered environment in Section \ref{sec:methods}, describe what the actual environment ends up looking like in Section \ref{sec:results}, and discuss the failures and hurdles I faced in Section \ref{sec:discussion}. I conclude in \ref{sec:conclusion} by discussing the improvements I would like to make in the future.

% Methods section
\section{Background}\label{sec:background}


\section{Methods}\label{sec:methods}
A development environment is set up on the device using Neo-vim with language-servers for Python and Rust. A VSCode-based development environment is set up on the Windows 10 machine for instances when a debugger is necessary. Git is installed on both machines for version control

% Results section
\section{Results}\label{sec:results}
\lipsum[6-7] % replace with your own text

% Discussion section
\section{Discussion}\label{sec:discussion}
\lipsum[8-9] % replace with your own text

% Conclusion section
\section{Conclusion}\label{sec:conclusion}
\lipsum[10] % replace with your own text

% References section
\begin{thebibliography}{9}
\bibitem{example1} 
Author A, Author B, and Author C. 
\textit{Title of the article}. 
Journal name, volume, issue, pages, year.

\bibitem{example2} 
Author D, Author E, and Author F. 
\textit{Title of the book}. 
Publisher, edition, year.
\end{thebibliography}

% Acknowledgments section
\section*{Acknowledgments}
\lipsum[11] % replace with your own text

% Appendices section
\appendix
\section{Appendix A}
\lipsum[12] % replace with your own text

\section{Appendix B}
\lipsum[13] % replace with your own text

% Figures and tables section
\section*{Figures and Tables}
% Example of including a figure
\begin{figure}[h]
\centering
\includegraphics[width=0.5\textwidth]{ferris_the_crab.png} % replace with your own image file
\caption{Example of a figure caption}
\label{fig:example}
\end{figure}

% Example of creating a table
\begin{table}[h]
\centering
\begin{tabular}{|c|c|c|}
\hline
A & B & C \\ \hline
1 & 2 & 3 \\ \hline
4 & 5 & 6 \\ \hline
\end{tabular}
\caption{Example of a table caption}
\label{tab:example}
\end{table}

\end{document}
